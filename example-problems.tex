\begin{question}[class=A]{2}
  \label{question:true-false-a}
  \TrueFalseInstructions \\
  \begin{minipage}{0.85\textwidth}
    True or False question version \GetVersionID.
  \end{minipage}%
  \begin{minipage}{0.15\textwidth}
    \begin{tasks}(1)
      \task[\choice] \ True
      \task[\correctchoice] \ False
    \end{tasks}
  \end{minipage}
\end{question}

\begin{question}[class=B]{2}
  \label{question:true-false-b}
  \TrueFalseInstructions \\
  \begin{minipage}{0.85\textwidth}
    True or False question version \GetVersionID.
  \end{minipage}%
  \begin{minipage}{0.15\textwidth}
    \begin{tasks}(1)
      \task[\correctchoice] \ False
      \task[\choice] \ True
    \end{tasks}
  \end{minipage}
\end{question}

\noindent\rule{\textwidth}{1pt}

\begin{question}[class=Z]{3}
  \label{question:multiple-choice}
  \MultipleChoiceInstructions \\

  Multiple Choice question.
  \begin{tasks}(4)
    \task[\choice] \ Choice 1
    \task[\correctchoice] \ Choice 2
    \task[\choice] \ Choice 3
    \task[\choice] \ Choice 4
  \end{tasks}
\end{question}

\noindent\rule{\textwidth}{1pt}

\begin{question}[class=Z]{3}
  \label{question:select-all}
  \SelectAllInstructions \\

  Select All question.
  \begin{tasks}(4)
    \task[\selectall] \ Choice 1
    \task[\correctselectall] \ Choice 2
    \task[\selectall] \ Choice 3
    \task[\selectall] \ Choice 4
  \end{tasks}
\end{question}

\noindent\rule{\textwidth}{1pt}

\begin{question}[class=Z]{3}
  \FillBlank{answer} Blank question
\end{question}

\noindent\rule{\textwidth}{1pt}

\begin{question}[class=Z]{3}
  \label{question:easy-problem}
  Easy problem with short work.
\end{question}
\begin{minipage}{0.70\textwidth}
  \begin{solution}
    Work.
  \end{solution}
\end{minipage}\hspace{\fill}%
\begin{minipage}{0.25\textwidth}
  \AnswerBox{Final Answer:}{Correct answer}
  \vspace{0.1in}
\end{minipage}

\noindent\rule{\textwidth}{1pt}

\begin{question}[class=Z]{6}
  \label{question:formula-problem}
  You want to start a savings account for a graduation trip that you wish to take in 5 years.
  Your plan is to deposit \(\$ 250\) each month for the first 4 years into an account paying interest at a rate of \(3.1\%\) per year compounded monthly.
  Then, for the last year you won't make any withdrawals or deposits.
  How much will you have saved in the account at the end of the 5 years?\\
  \FormulaInstructions~\RoundUnitsInstructions
  \begin{tasks}(7)
    \task[\selectall] \ A
    \task[\selectall] \ B
    \task[\correctselectall] \ C
    \task[\selectall] \ D
    \task[\selectall] \ E
    \task[\correctselectall] \ F
    \task[\selectall] \ G
  \end{tasks}
  \mrubric{Margin rubric}
\end{question}
\begin{minipage}{0.70\textwidth}
  \begin{solution}
    \begin{align*}
      S = 250 \left[ \dfrac{\left(1 + \frac{0.031}{12} \right)^{12\cdot4} - 1}{\frac{0.031}{12}} \right] &= 12,758.21477 \\
      A = 12,758.21477 \left( 1 + \dfrac{0.031}{12} \right)^{12\cdot1} &= 13,159.38756
    \end{align*}
  \end{solution}
  \rubric{Rubric info if needed}
\end{minipage}\hspace{\fill}%
\begin{minipage}{0.25\textwidth}
  \vspace{1.5in}
  \AnswerBox{Final Answer:}{\(\$13,159.39\)}
  \vspace{0.1in}
\end{minipage}

\noindent\rule{\textwidth}{1pt}


\begin{question}[class=Z]{6}
  \label{question:tvm-table}
  You are buying a house for \(\$300,000\) and you secured a loan to finance the remaining cost after a down payment of \(10\%\).
  The loan will be paid over a period of 20 years and it accrues interest at a rate of \(6.5\%\) per year compounded monthly.
  How much will you pay each month? \\
  \TVMInstructions \\

  \TVMTable{\(20\cdot12=240\)}{alpha}{6.5}{0}{270,000}{12 or monthly}
\end{question}

\noindent\rule{\textwidth}{1pt}

\begin{question}[class=Z]{6}
  \label{question:amortization-table}
    In order to buy a car Larry financed \(\$100,000\) through a 10 year loan that charges interest at a rate of \(5.7\%\) per year compounded monthly. \\
    \AmortizationInstructions \\

    \begin{amortization}
    0 & -- & -- & -- & \(100,000\) \\
    \hline
    1 & \(475.00\) & \(1,095.20\)  & \(620.20\)  & \(99,379.80\)  \\
    \hline
    2 & \answer{\(472.05\)} & \answer{\(1,095.20\)} & \answer{\(623.16\)} & \answer{\(98,756.65\)} \\
    \end{amortization}
\end{question}

\noindent\rule{\textwidth}{1pt}

\begin{question}[class=Z]{6}
  \label{question:setup-system}
  Setup variables and system of equations.\\[0.2in]
  \begin{minipage}{0.45\textwidth}
    \AnswerBox[1.0in]{Define Variables:}{Correct answer}
  \end{minipage}\hspace{\fill}%
  \begin{minipage}{0.45\textwidth}
    \AnswerBox[1.0in]{System of Equations:}{Correct answer}
  \end{minipage}
\end{question}

\noindent\rule{\textwidth}{1pt}

\newpage

\begin{question}[class=Z]{6}
  \label{question:graph-question}
  Problem with large graph.\\
  \begin{minipage}{0.45\textwidth}
  \end{minipage}\hspace{\fill}%
  \begin{minipage}{0.5\textwidth}
    \begin{center}
      \begin{largegraph}
        \addplot+[<->, samples=100,] {x} node[pos=1,above] {\(L_{1}\)};
        \addplot+[<->,dashed, samples=100] {-2*x} node[pos=0,above] {\(L_{2}\)};
        \pgfplotsset{cycle list shift=-2}
        \addplot+[samples=100, draw=none, fill, fill opacity=0.2] {x} |- cycle;
        \addplot+[samples=100, draw=none, fill, fill opacity=0.2] {-2*x} -- (\pgfkeysvalueof{/pgfplots/xmin},\pgfkeysvalueof{/pgfplots/ymin}) |- cycle;
      \end{largegraph}
    \end{center}
  \end{minipage}
\end{question}

\noindent\rule{\textwidth}{1pt}


\begin{question}[class=Z]{6}
  \label{question:q-one-graph-question}
  Problem with large quadrant 1 graph.\\
  \begin{minipage}{0.45\textwidth}
  \end{minipage}\hspace{\fill}%
  \begin{minipage}{0.5\textwidth}
    \begin{center}
      \begin{largegraphqone}
        \addplot+[<->, samples=100,] {x} node[pos=1,above] {\(L_{1}\)};
        \addplot+[<->,dashed, samples=100] {-2*x + 6} node[pos=0,above] {\(L_{2}\)};
        \PrintSolutionsT{
          \pgfplotsset{cycle list shift=-2}
          \addplot+[samples=100, draw=none, fill, fill opacity=0.2] {x} |- cycle;
          \addplot+[samples=100, draw=none, fill, fill opacity=0.2] {-2*x + 6} -- (\pgfkeysvalueof{/pgfplots/xmin},\pgfkeysvalueof{/pgfplots/ymin}) |- cycle;
        }
      \end{largegraphqone}
    \end{center}
  \end{minipage}
\end{question}

\noindent\rule{\textwidth}{1pt}


\begin{question}[class=Z]
  \label{question:small-graph-question}
  Multipart problem
  \begin{enumerate}[label = \textbf{\alph*)}]
    \item{} [\addpoints{4}] Intercept Problem \\
      \intercepts{(-1,0)}{(0,1)}{(2,0)}{(0,1)}
    \item{} [\addpoints{3}] Problem with small graphs\\
      \begin{tasks}(4)
        \task[\choice] \ \(\begin{smallgraph} \addplot {x + 1}; \addplot {2*x + 2}; \end{smallgraph}\)
        \task[\correctchoice] \ \(\begin{smallgraph} \addplot {x + 1}; \addplot {-1/2*x + 1}; \end{smallgraph}\)
        \task[\choice] \ \(\begin{smallgraph} \addplot {-x + 1}; \addplot {-1/2*x + 1}; \end{smallgraph}\)
        \task[\choice] \ \(\begin{smallgraph} \addplot {-x + 1}; \addplot {2*x + 2}; \end{smallgraph}\)
      \end{tasks}
  \end{enumerate}
\end{question}

\noindent\rule{\textwidth}{1pt}


\begin{question}[class=Z]{6}
  Problem with tableau.\\
  \begin{align*}
    \text{Maximize: } \  \ P = 5x &+ 6y \\
    \text{Subject to: } \  \ x + 2y &\le 7 \\
    3x + 4y &\le 8 \\
    x \le 0 \ \text{,}& \ y \le 0
  \end{align*}
  \[
  \begin{tableau}{5}
    x & y & u & v & P & \text{Constants} \\
    \hline
    \circled{1} & 2 & 1 & 0 & 0 & 7 \\
    3 & \circledIF{4} & 0 & 1 & 0 & 8 \\
    \hline
    -5 & -6 & 0 & 0 & 1 & 0
  \end{tableau}
  \]
\end{question}
